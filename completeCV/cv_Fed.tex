\documentclass[12pt,a4paper,sans]{moderncv}        %

% load packages
\moderncvstyle{classic}
%\renewcommand*{\namefont}{\fontsize{18}{18}\mdseries\upshape}
\moderncvcolor{grey}
\usepackage[scale=0.8]{geometry}
\usepackage{multibib}
\newcites{ipr,pr,pend,nr,dp,data}{{InvitedPresentations},{conference},{inprep},{NonRefereed},{departmental}, {datasets}}


\newenvironment{myitemize}%
  {\begin{list}{}{%
    \setlength{\labelwidth}{0pt}
    \setlength{\leftmargin}{\hintscolumnwidth+\separatorcolumnwidth}
    }}%
  {\end{list}}

\newcommand*{\cventrylong}[7][.25em]{%
  \begin{tabular}{@{}p{\hintscolumnwidth}@{\hspace{\separatorcolumnwidth}}p{\maincolumnwidth}@{}}%
    \raggedleft\hintstyle{#2} &{%
        {\bfseries#3}%
        \ifthenelse{\equal{#4}{}}{}{, {\slshape#4}}%
        \ifthenelse{\equal{#5}{}}{}{, #5}%
        \ifthenelse{\equal{#6}{}}{}{, #6}%
    }%
  \end{tabular}%
  \vspace*{-\baselineskip}
  {\vspace{-.25em}
   \small\begin{myitemize}\item #7\end{myitemize}}%
  \par\addvspace{#1}%
}


%% FOR VITAE
\usepackage{mathpazo}
\renewcommand{\sfdefault}{ppl}
%\newcommand{\size}[2]{{\fontsize{#1}{0}\selectfont#2}}

%\renewcommand*\namefont{\fontfamily{ppl}\fontsize{16}{16}\selectfont}
%\renewcommand*\titlefont{\fontfamily{ppl}\fontsize{12}{12}\selectfont}
%\renewcommand*\addressfont{\fontfamily{ppl}\selectfont}
%\renewcommand*\sectionfont{\fontfamily{ppl}\fontsize{12}{12}\selectfont}
%\renewcommand{\sfdefault}{ppl}

% bold name in citations or wherever it appears
\def\FormatName#1{%
  \def\myname{Levine R}%
  \edef\name{#1}%
  \ifx\name\myname
    \textbf{#1}%
  \else
    #1%
  \fi
}
    %
\name{Robert}{Levine}

%%%---- personal details----%%%
%\title{CV}
\address{1013 NE 40th St.}{Seattle, WA 98105}{}%
\phone[home]{+1~(914)~391~9490}
\email{LevineR@uw.edu}
%\homepage{RobertMLevine.com}
%\social[twitter]{bobertshmobert}
%\social[github]{leviner}

\begin{document}

\makecvtitle
%%%% Set last updated footnote page 1
%\begin{tikzpicture}[remember picture,overlay]
%	\node[anchor=south, yshift=0.8cm, xshift=6cm] at (current page.south) {\textcolor{color1}{\textit{Last updated \today}}};
%\end{tikzpicture}

\vspace{-10mm}

%%%------------------------EDUCATION-----------------------%%%
%\begin{center}
%    \size{14}{\textbf{VITAE}}
%\end{center}

\section{Education}
\cventry{12/2021}{PhD, Oceanography}{University of Washington}{Seattle, WA}{}{}{}
\cventry{8/2014}{MS, Geological Sciences}{Cornell University}{Ithaca, NY}{}{}{}{}
\cventry{5/2012}{BS, Interdisciplinary Studies in Agriculture and Life Sciences}{Cornell \mbox{University}}{Ithaca, NY}{}{}{}{}
%\section{Summary}
%Biological oceanographer with 10+ years experience in zooplankton and fisheries acoustics. Experience with autonomous and remote echosounder systems, testing and producing data processing pipelines for new instrumentation, and integration of acoustic data with physical and chemical oceanographic data and model predictions. More than 275 days at-sea as a scientist or instrument technician. Special interest in application of broadband acoustics for fisheries applications.\vspace{-4mm} \\

%%%------------------------EXPERIENCE-----------------------%%%
%\section{Academic Positions}
%\cventry{2016--\textit{present}}{Graduate Research Assistant}{University of Washington School of Oceanography}{Seattle, WA}{}{}

\section{Professional Experience}
\cventry{1/2022--Present}{Postdoctoral Scholar}{UW Applied Physics Laboratory}{1013 NE 40th St Seattle, WA}{}{Hours per week: 40, Salary: \$60,000/year \\Supervisor: Christopher Bassett (cbassett@uw.edu, 206-543-1263)
\vspace{2mm} \\ Postdoctoral Scholar in the Ocean Engineering department. Conducting analysis as part of two research projects in collaboration with the NOAA Alaska Fisheries Science Center:
\begin{itemize}
    \item Estimating the flux of walleye pollock between US and Russian sectors in the northwestern Bering Sea using data from a one-year deployment of seafloor mounted autonomous echosounders. Integrating observations with environmental data and model predictions of currents and environmental conditions.
    \item Analysis of discrepancies between broadband and narrowband volume scattering collected using Simrad EK80 echosounders. The goal of this work is to identify potential underlying causes of variability in integrated volume scattering as a result of differences in signal processing and provide guidance for use of broadband signals for use in acoustic-trawl surveys.
\end{itemize}
\vspace{2mm} Both of these projects include the development of both MATLAB- and Python-based tools for parsing and analyzing broadband data, in particular, calculating and classifying spectra from both bulk scattering and individual targets based on scattering models.}

\cventrylong{9/2016--\\12/2021}{Graduate Research Assistant}{UW School of Oceanography}{1503 NE Boat St Seattle, WA}{}{Hours per week: 20, Salary: \$32,000/year \\Supervisor: Daniel Grünbaum (random@uw.edu, 206-221-6594)
\vspace{2mm} \\Conducted fisheries acoustics research as part of the North Pacific Research Board Arctic Integrated Ecosystem Research Program.
\begin{itemize}
    \item Both assisted and led acoustic-trawl component of interdisciplinary surveys of the eastern Chukchi Sea, including assisting in the development of sampling procedures, leading echosounder calibrations, trawl operations, data analyses, and publication of findings.
    \item Co-developed and managed autonomous surface vessel acoustic surveys of the eastern Chukchi Sea. Led data analysis, reporting, and publication of findings.
    \item Co-developed and managed the deployment of autonomous echosounder moorings. Assisted in the design, construction, deployment, and recovery of moorings. Led data analysis, including the analysis and integration of acoustic Doppler current profiler datasets with fisheries acoustic data.
\end{itemize}
}

\cventry{6/2014--\\8/2016}{Scientist III}{Ocean Associates, Incorporated}{4007 N Abingdon St Arlington, VA}{}{Hours per week: 40, Salary: \$30/hour \\Supervisor: Peter Milone (PeterMilone@oceanassoc.com)
\vspace{2mm} \\Contracted to the Midwater Assessment and Conservation Engineering Program at the NOAA Alaska Fisheries Science Center in support of survey operations and moored echosounder projects, including:
\begin{itemize}
    \item Assistance in the development, operation, and data analysis of moored echosounders for deployment in the Gulf of Alaska to investigate the timing of pollock spawning migration.
\item Development of MATLAB-based methods for system calibration, system stability testing, data quality testing, data processing, and final analysis of data collected during 3-month deployment.
\item Construction and testing of echosounder housing and recovery system.
\end{itemize}
\vspace{2mm} \\Assisted in the development of survey analysis software, including:
\begin{itemize}
    \item Development of Python-based software for batch exportation of acoustic data and exploration tools for quality control and reporting of acoustic-trawl survey data.
\end{itemize}
\vspace{2mm} \\Served as a scientist for over 150 days at sea for gear trials and pollock surveys, including:
\begin{itemize}
    \item Assisted in the calibration of ship-mounted narrowband and multibeam echosounders.
    \item Processed net catches to determine species composition, counts, lengths, weights, ages, and sexes of sampled populations.
\end{itemize}
}

\cventry{1/2012--\\5/2014}{Guest Student}{Woods Hole Oceanographic Institution}{86 Water St Woods Hole, MA}{}{Hours per week 40, Unpaid\\Supervisor: Gareth Lawson
\vspace{2mm} \\ Initial appointment as an undergraduate student, extended to facilitate research associated with MS degree under co-advisement of Gareth Lawson (WHOI) and Charles Greene (Cornell University). Analysis of zooplankton acoustic surveys, including:
\begin{itemize}
\item Participation in zooplankton acoustic surveys and dockside instrument testing, including calibration, data collection, processing, and analyses of both narrowband and broadband echosounders.
\item Analysis of zooplankton multifrequency scattering response using forward calculations to estimate the contribution of scattering from zooplankton based on net tows. Production of a distributable MATLAB-based software package and technical manual for forward calculations.
\item Taxonomy and silhouette analyses of zooplankton with a focus on euphausiid species identification.
\item Assembly, maintenance, and operation of multi-net systems for zooplankton collection.
\end{itemize}}

  


%%%------------------------PUBLICATIONS-----------------------%%%
\newpage
\section{Publications}

%% Peer reivewed %%
\renewcommand*{\bibliographyhead}[1]{\subsection{Peer-reviewed}}
\subsection{\textit{Published}} % UNCOMMENT IF NEEDED 
\nocite{*}
\bibliographystyle{cv}
\bibliography{publications}

%% In prep %%
% UNCOMMENT IF NEEDED 
\subsection{\textit{Upcoming}}% UNCOMMENT IF NEEDED 
\renewcommand*{\bibliographyhead}[1]{\subsection{In Preparation}}
\nocitepend{*}
\bibliographystylepend{cv}
\bibliographypend{inprep}

%% Non-refereed %%
\renewcommand*{\bibliographyhead}[1]{\subsection{Non-refereed}}
\subsection{\textit{Non-refereed}}
\nocitenr{*}
\bibliographystylenr{cv}
\bibliographynr{nonrefereed}

%%%------------------------INVITEDPRESENTATIONS-----------------------%%%
\section{Invited Presentations}
\renewcommand*{\bibliographyhead}[1]{}
\nociteipr{*}
\bibliographystyleipr{cv}
\bibliographyipr{invitedpresentations}

\renewcommand*{\bibliographyhead}[1]{\subsection{Departmental Seminars}}
\subsection{\textit{Departmental Seminars}}
\nocitedp{*}
\bibliographystyledp{cv}
\bibliographydp{departmental}

%%%------------------------PRESENTATIONS-----------------------%%%
\section{Conference Abstracts}
\renewcommand*{\bibliographyhead}[1]{}
\nocitepr{*}
\bibliographystylepr{cv}
\bibliographypr{conference}

%%%------------------------TEACHING-----------------------%%%
\section{Teaching Experience}
\cventry{Winter 2022}{Guest Lecturer}{Ocean Technology}{UW School of Oceanography}{}{Lecture on field applications of active acoustics for fisheries oceanography}
\cventry{Spring 2021}{Guest Lecturer}{Oceanography}{UC Davis Earth and Planetary Sciences}{}{Lecture on Arctic circulation and ecological impact of sea ice reduction.}
\cventry{Spring 2021}{Guest Lecturer}{Arctic Change}{UW School of Oceanography}{}{Lecture on role of sea ice in Pacific Arctic fisheries.}
\cventry{Winter 2019}{Guest Lecturer}{Foundations of Ocean Sensors}{UW School of Oceanography}{}{Activity to measure speed of sound and conduct bathymetric survey using student-built sensors.}
\cventry{Spring 2018}{Guest Lecturer}{Marine Zooplankton Ecology}{UW School of Oceanography}{}{Lecture on application of active acoustics for zooplankton ecology.}
\cventry{Winter 2018}{Teaching Assistant}{Foundations of Ocean Sensors}{UW School of Oceanography}{}{Assisted with course activities and designed module using ultrasonic acoustic sensors.}
\cventry{Fall 2013}{Teaching Assistant}{Introductory Oceanography}{Cornell University}{}{Assisted with course activities and designed module using ultrasonic acoustic sensors.}
\cventry{Summer 2013}{Teaching Assistant}{Marine Bioacoustics Workshop}{Friday Harbor Laboratories}{}{Classroom and field assistant, guest lecturer for forward calculations and vessel calibration.}
\cventry{Summer 2013}{Teaching Assistant}{Satellite Remote Sensing Training Program}{Cornell University}{}{Assisted with computer-based exercises using NASA SeaDAS and Python.}
\cventry{Fall 2012}{Teaching Assistant}{Introductory Oceanography}{Cornell University}{}{Designed and led laboratory sections, assisted with design and grading of lecture component.}
\cventry{Summer 2011}{Teaching Assistant}{Satellite Remote Sensing Training Program}{Cornell University}{}{Assisted with computer-based exercises using NASA SeaDAS and IDL.}

%%%------------------------GRANTS-----------------------%%%
\section{Funding and Grants}
\cvitem{2020}{Diversity Funding Award (awarded to PublicSensors), \textit{UW College of the Environment}}
\cvitem{2014}{Graduate Conference Grant, \textit{Cornell University}}

%%%------------------------AWARDS-----------------------%%%
\section{Awards and Honors}
\cvitem{2021}{Best Student Oral Presentation \textit{Alaska Marine Science Symposium}}
\cvitem{2020}{Best Student Poster Presentation, \textit{Alaska Marine Science Symposium}}
\cvitem{2018}{Science Communication Fellow, \textit{Pacific Science Center}}
\cvitem{2014}{Bryan Isaacks Excellence in Teaching Award, \textit{Cornell University}}


%%%------------------------PROFESSIONAL DEVELOPMENT -----------------------%%%
\section{Job Related Training}
\cvitem{2021}{XSEDE HPC Workshop: Big Data and Machine Learning. Two-day interactive workshop on tools for machine learning implementation on HPC clusters using Spark and Hadoop. Pittsburgh Supercomputing Center (virtual).}{}
\cvitem{2020}{OceanHackWeek 2020. One-week training and project-based workshop, contributed to the development of open-source automated ship track segmentor. University of Washington, WA (virtual). University of Washington, WA}
\cvitem{2019}{Principles and Methods of Broadband/Wideband Technologies ICES Training Course. One-week shipboard training in principles of broadband acoustics and applications for fisheries research. Bergen, Norway}
\cvitem{2019}{OOI Early Career Scientist Interdisciplinary Workshop. Co-developed interdisciplinary research project using observatory data to detect ocean impacts of atmospheric anomalies. Washington, D.C.}
\cvitem{2018}{OOI Early Career Biology Data Workshop. One-week workshop in data processing pipeline development, quality control and assessment using Ocean Observatory Initiative data. Rutgers University, NJ}
\cvitem{2018}{UW Cabled Array Hack Week. One-week training and project-based workshop, developed open source toolkit for calculating volume backscatter from OOI ADCPs.  University of Washington, WA}
\cvitem{2011}{Bioacoustical Oceanography Workshop. Six-week training course on fisheries and marine mammal bioacoustics. Co-conducted a project on the limitations of non-negative least squares multifrequency inversion method for estimating abundance of zooplankton. Friday Harbor Laboratories, WA}

%%%------------------------Outreach -----------------------%%%
\section{Outreach}
\cvitem{2020--\textit{present}}{Program Developer, \httplink{PublicSensors.org} \textit{Educational initiative providing kits and instruction for hands-on learning with microcontroller-based environmental sensors. Lead K-12 student workshops and teacher professional development programs.}}
\cvitem{2020}{Meet a Scientist, Pacific Science Center Camps for Curious Minds \textit{Present and answer questions on underwater acoustics and Arctic ecology.}}
\cvitem{2018--2020}{Meet a Polar Scientist, Pacific Science Center \textit{Present ecology and dynamics of the Pacific Arctic as part of a live planetarium show titled "Pole to Pole".}}
\cvitem{2019}{Co-Program Leader, UW Aquatic Sciences Open House \textit{Table activity demonstrating microcontroller-based marine sensors.}}
\cvitem{2018}{Co-Program Leader, Pacific Science Center Polar Science Weekend \textit{Hands-on activities exploring microbial life in sea ice.}}
\cvitem{2017--2018}{Facilitator, UW SeaState \textit{Conducted sensor building programs in high school chemistry and oceanography classes in Sequim and South Kitsap, WA in which the students build and use photospectrometers to explore changes in pH.}}
%%%------------------------SERVICE-----------------------%%%
\section{ Service}\subsection{University}
\cvitem{2020}{Series Organizer, UW College of the Environment Beyond Academia Speaker Series}
\cvitem{2019--2020}{UW College of the Environment Student Advisory Council Graduate Senate Liason}
\cvitem{2018--2020}{Elections Committee, {UW Graduate and Professional Student Senate}}
\cvitem{2017--2020}{School of Oceanography Senator, {UW Graduate and Professional Student Senate}}
\cvitem{2018--2019}{Officer, {UW Academic and Recreational Graduate Oceanographers}}

\subsection{Conferences}
\cvitem{2019}{Session chair of ``Arctic Fishes and Fish Habitat'' at Alaska Marine Science Symposium, Anchorage, AK.}
\subsection{Reviewer}
\cvitem{}{\textit{Oceanography, Polar Biology, Frontiers in Marine Science}}

%\subsection{Reviewer}
%%%------------------------SERVICE-----------------------%%%
\section{Professional Affiliations}
\cvitem{}{Association for the Sciences of Limnology and Oceanography}
\cvitem{}{Acoustical Society of America}
\cvitem{}{ICES Working Group on Fisheries Acoustics, Science and Technology}


%%%------------------------Field Experience-----------------------%%%
\section{Field Experience}
\cventry{2019}{R/V \textit{G.O. Sars}}{ICES Broadband Workshop}{7 days}{Veafjorden}{Student.}
\cventry{2019}{R/V \textit{Ocean Starr}}{NPRB Arctic IERP}{40 days}{Chukchi and Beaufort Seas}{Acoustic-trawl operations co-lead, acoustic mooring recovery lead.}
\cventry{2018}{USCGC \textit{Healy}}{DBO-NCIS Program}{19 days}{Chukchi and Beaufort Seas}{Methot sampling lead, acoustic mooring deployment/recovery lead.}
\cventry{2017}{R/V \textit{Ocean Starr}}{NPRB Arctic IERP}{36 days}{Chukchi and Beaufort Seas}{Acoustic-trawl operations co-lead.}
\cventry{2016}{NOAAS \textit{Oscar Dyson}}{Pollock Acoustic-Trawl Survey}{40 days}{Bering Sea}{Scientist/ET.}
\cventry{2015}{NOAAS \textit{Oscar Dyson}}{Pollock Acoustic-Trawl Survey}{77 days}{Gulf of Alaska. Scientist}{}
\cventry{2014}{NOAAS \textit{Oscar Dyson}}{Pollock Acoustic-Trawl Survey}{38 days}{Bering Sea}{Scientist.}
\cventry{2014}{R/V \textit{Tioga}}{WHOI OAPS}{4 days}{Gulf of Maine}{Scientist.}
\cventry{2013}{R/V\textit{Centennial}}{FHL Bioacoustics Workshop}{2 days}{Puget Sound}{Instructor, acoustic calibration lead.}
\cventry{2012}{R/V \textit{New Horizon}}{WHOI OAPS}{13 days}{Northeast Pacific}{Acoustic technician.}
\cventry{2011}{R/V \textit{Centennial}}{FHL Bioacoustics Workshop}{2 days}{Puget Sound}{Student.}

%%%------------------------Other-----------------------%%%
\section{Additional Information}
\cvitem{}{Citizen of the United States of America}{}{}
\cvitem{}{Registered for Selective Service}{}{}
%\cvitem{\textbf{Certifications}}{}
%\cvitem{\textbf{Computation}}{}
%\cvitem{\textbf{Languages}}{}

%%%--------------------Authored Datasets-----------------------%%%
%\section{Authored Software and Datasets}
%\renewcommand*{\bibliographyhead}[1]{\subsection{datasets}}
%\nocitedata{*}
%\bibliographystyledata{cv}
%\bibliographydata{datasets}

\end{document}
