\documentclass[12pt,a4paper,sans]{moderncv}        %

% load packages
\moderncvstyle{classic}
%\renewcommand*{\namefont}{\fontsize{18}{18}\mdseries\upshape}
\moderncvcolor{grey}
\usepackage[scale=0.8]{geometry}
\usepackage{multibib}
\newcites{ipr,pr,pend,nr,dp,data}{{InvitedPresentations},{conference},{inprep},{NonRefereed},{departmental}, {datasets}}

\name{Robert}{Levine}

% Use this to include the Signature.PNG graphic in the letter closing
\renewcommand*{\makeletterclosing}{%
  \vspace{3.4mm}
  \lettertextstyle{Sincerely,} \\\\
  \includegraphics[width=6cm]{Signature.PNG}
  \\
  \letternamestyle{Robert Levine}    }

%% FOR VITAE
\usepackage{mathpazo}
\renewcommand{\sfdefault}{ppl}


%%%---- personal details----%%%
%\title{CV}
\address{Robert Levine \\ 1013 NE 40th St.\\}{Seattle, WA 98105}{}%
\phone[home]{+1~(914)~391~9490}
\email{LevineR@uw.edu}
%\homepage{RobertMLevine.com}
%\social[twitter]{bobertshmobert}
%\social[github]{leviner}

\begin{document}

\null\hfill\textbf{Robert Levine}\\
\null\hfill  \textit{1013 NE 40th St.}\\
\null\hfill  \textit{Seattle, WA 98105}\\
\null\hfill \textit{+1 (914) 391 9490}\\
\null\hfill \textit{ LevineR@uw.edu}\\
8 May, 2022\\\\
Midwater Assessment and Conservation Engineering Program\\
NOAA Alaska Fisheries Science Center\\
7600 Sand Point Way\\
Seattle, WA 98115\\\\
To Whom it May Concern,\\

I am writing to express my interest in the role of Research Physical Scientist. I will graduate with my PhD in Oceanography from University of Washington this December, and believe that my 9+ years of experience as a researcher and technician specializing in remote oceanographic instrumentation will allow me to contribute to mission of Saildrone to provide high quality data across ocean and atmospheric disciplines.\\

%\lettersection{Why Saildrone?}
My interest in remote sensing and autonomous platforms is what led me to pursue my PhD in oceanography. As a contractor at the National Oceanographic and Atmospheric Administration, I worked alongside instrument engineers on the development of a moored autonomous echosounder, including preliminary instrument testing and data quality assessment in order to ensure recovery of quality data to support the intended scientific mission. Throughout this work, I was also providing survey support, responsible for the operation and maintenance of instrumentation at sea which exposed me to a wide range of oceanographic equipment and data types. \\

Throughout my PhD, these skills allowed me to focus on the integration of interdisciplinary data sets. Developing skills in data formatting and processing, my research was based on collecting and interpreting temperature, salinity, current, and water column sonar datasets to explore biophysical interactions. As a member of the Ocean Observatory Initiative Early Career Scientists community, I participated in data assessment workshops and led an interdisciplinary investigation of storm impacts on the biology, physics, and chemistry of the water column which required knowledge of a variety of unique sensors and the programming skills to acquire, asses, integrate, and analyze diverse datasets.  

\\
\makeletterclosing
\end{document}

%%Collection and analysis of data from fisheries echosounders, Acoustic Doppler Current Profilers (ADCPs), imaging sonars, multibeam sonars, sidescan sonars, or sub-bottom profilers; Development and application of backscattering or species classification models applicable to remote sensing of marine organisms; Proficiency in the use of programming languages including Matlab, Python, and/or R for data- and signal-processing
